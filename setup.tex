% PAGE DIMENSIONS
\usepackage{geometry}
\geometry{a4paper,margin=3.5cm}

% PACKAGES
\usepackage{graphicx} %support the \includegraphics command and options
\usepackage{fancyhdr} % Headers/footers (should be set AFTER setting up the page geometry and before hyperref)
\usepackage{color}
\usepackage{eso-pic} %background pictures
\usepackage[latin1]{inputenc}
%\usepackage[danish]{babel}
\usepackage{csquotes}
\usepackage{booktabs} % for much better looking tables
\usepackage{tabularx}
\usepackage{slashbox}
\usepackage{array} % for better arrays (eg matrices) in maths
\usepackage{paralist} % very flexible & customisable lists (eg. enumerate/itemize, etc.)
\usepackage{verbatim} % adds environment for commenting out blocks of text & for better verbatim
\usepackage{alltt} %improved verbatim
\usepackage{titlesec} %to modify \chapter, \section, etc appearance
\usepackage{hyperref} %til links, email etc - giver ogs bookmarks i pdf filen
%\usepackage{dirtree} %directory trees
\usepackage{subfig} %make it possible to include more than one captioned figure/table in a single float
\usepackage{float} %float positioning etc
\usepackage{amsmath,amsfonts,amssymb,amsthm} %AMS' packages for symbols, theorems etc.
\usepackage{xfrac} %for nicefrac{}{}
\usepackage{wrapfig}
\usepackage{multicol}
\usepackage{footnote}
\usepackage{perpage}
\usepackage{ctable}
\usepackage[intoc]{nomencl} %for abbreviations list
\usepackage{listings} %for source code listings
\usepackage{marginnote} %Margin notes - use \marginnote{}
%\usepackage[parfill]{parskip} %Activate to begin paragraphs with an empty line rather than an indent

% marginnote package options
\renewcommand*{\marginfont}{\color{red}\sffamily} %red, sans-serif
\reversemarginpar %margin notes on left side

% makes footnotes in tables possible (perpage package)
\MakePerPage{footnote}
\makesavenoteenv{tabular}

% listings package settings
\lstset{
	language=c++,
	numbers=left,
	numberstyle=\scriptsize,
	numbersep=6pt,
	captionpos=b,
	tabsize=4,
	basicstyle=\footnotesize,
	breaklines=true,% sets automatic line breaking
	escapeinside={(*�}{�*)},
}
%\renewcommand*\lstlistingname{Code}


% hyperref package settings
\hypersetup{
    unicode=true,          % non-Latin characters in Acrobat�s bookmarks
    pdftoolbar=true,        % show Acrobat�s toolbar?
    pdfmenubar=true,        % show Acrobat�s menu?
    pdffitwindow=false,     % window fit to page when opened
    pdfstartview={FitH},    % fit page to the window Horizontal/Vertical
    pdftitle={TITLE},    % title
    pdfauthor={Alexander Adelholm Brandbyge, Frederik Hagelskj�r, Rudi Hansen, Leon Bonde Larsen, Kent Stark Olsen, Kim Lindberg Schwaner},% author
    pdfsubject={SUBJECT},   % subject of the document
    pdfkeywords={DTMF} {keyword2} {SDU}, % list of keywords
    pdfnewwindow=true,      % links in new window
    colorlinks=false,       % false: boxed links; true: colored links
    linkcolor=red,          % color of internal links
    citecolor=green,        % color of links to bibliography
    filecolor=magenta,      % color of file links
    urlcolor=cyan,           % color of external links
    plainpages=false
}

% Bibliography appearance
\usepackage[style=authortitle-icomp,natbib=true,sortcites=true,block=space,backend=bibtex8]{biblatex}
\setlength{\bibparsep}{10pt}
\bibliography{content/bibliography}

% Headers and footers
\pagestyle{fancy} % options: empty , plain , fancy
\setlength{\headheight}{15pt}
\lhead{\nouppercase{\rightmark}}%\lhead{\nouppercase{\leftmark}}
\chead{}
\rhead{}
\lfoot{}
\cfoot{}
\rfoot{\thepage}

\fancypagestyle{plain}{% Using the plain-style to be able number pages, but with an empty header! (using the report document class makes this nearly obsolete)
 \fancyhf{}
 \renewcommand{\headrulewidth}{0pt}
 \fancyfoot[RO]{\thepage}
}

%%% Contents appearance
\usepackage[]{tocbibind} % Put the bibliography in the ToC (Opts: nottoc,notlof,notlot)
\setcounter{tocdepth}{3} % set how many levels the table of contents displays. default=3
\usepackage[titles,subfigure]{tocloft} % Alter the style of the Table of Contents

% \includegraphics default folder
\graphicspath{{content/graphics/}}

% Number by section
%\numberwithin{equation}{section}
%\numberwithin{figure}{section}
%\numberwithin{table}{section}

% Paragraphs (handled by the parskip package currently?)
%\setlength{\parindent}{0pt}
%\setlength{\parskip}{2ex plus 0.5ex minus 0.2ex}

% Float positioning control
\setcounter{topnumber}{2}
\setcounter{bottomnumber}{2}
\setcounter{totalnumber}{3}
\renewcommand{\topfraction}{0.85}
\renewcommand{\bottomfraction}{0.85}
\renewcommand{\textfraction}{0.15}
\renewcommand{\floatpagefraction}{0.4}

% Header colour
\definecolor{FrontpageHeadingColor}{RGB}{5,5,60}%Heading colour definition

% Chapter name formatting
\titleformat
  {\chapter}%command
  [display]%shape
  {\normalfont\huge\bfseries}%format
  {\normalfont\Large\scshape\chaptertitlename\ \huge\thechapter}%label
  {10pt}%sep
  {\Huge}%before

% make nomenclature and change its heading/toc text (see nomencl package)
\makenomenclature
\renewcommand{\nomname}{Abbreviations}
%
%  * Pass this line to MakeIndex:
%    %bm.nlo -s nomencl.ist -o %bm.nls
%
%  * Use \nomenclature{abbr}{discriptive text} to add an entry to the 
%    abbreviations list (best done where the abbreviation first occurs in the text)
%  * Example:
%    \nomenclature{ADHD}{Attention Deficit Hyperactivity Disorder}
