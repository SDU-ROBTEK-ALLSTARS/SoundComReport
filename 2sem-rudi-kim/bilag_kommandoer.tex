\section{Kommandoer}\label{sec:bilag_kommandoer}
I det, der er implementeret en softwarekommunikationsprotokol, er der en given m�ngde kommandoer bilen kan reagere p�. Samtidig er der liges� et antal mulige svar. For at holde styr p� dem alle findes i Tabel \ref{tab:kommandoer} en oversigt over kommandoer.

\begin{savenotes}
\begin{table}[htb]
	\begin{center}
	\begin{tabular}{l|l|l}
	
	TYPE									& KOMMANDO																					& DATA																		\\
	\hline
	\verb@0x55@ (Set) 		& \verb@0x01@ (S�t low(adresse\footnote{test}))			& \verb@0x00 - 0xFF@ (Hastighedsniveau)		\\
												& \verb@0x09@ (S�t high(adresse\footnote{test}))		& \verb@0x00 - 0xFF@ (Hastighedsniveau)		\\
												& \verb@0x09@ (K�r bagl�ns)													& \verb@0x00 - 0xFF@ (Hastighedsniveau)		\\
												& \verb@0x09@ (K�r bagl�ns)													& \verb@0x00 - 0xFF@ (Hastighedsniveau)		\\
			 									& \verb@0x10@ (K�r forl�ns)													& \verb@0x00 - 0xFF@ (Hastighedsniveau)		\\
			 									& \verb@0x11@ (Stop)																& \verb@-@																\\
			 									& \verb@0x12@ (\verb@MODE@=Auto) 										& \verb@-@																\\
												& \verb@0x12@ (\verb@MODE@=Default)									& \verb@-@																\\
	\hline
	
	\verb@0xAA@ (Get)			& \verb@0x12@ (\verb@MODE@)													& \verb@0x??@ (\verb@MODE@ register)			\\
	
	\hline
		
	\verb@0xBB@ (Reply)		& \verb@0x10@ (Error)																& \verb@0x00 - 0xFF@ (Fejlkode)						\\
												& \verb@0x12@ (\verb@MODE@)													& \verb@0x??@ (\verb@MODE@ register)			\\

	\end{tabular}
	\end{center}
	\caption{Telegramkommandoer}
	\label{tab:kommandoer} %%ref
\end{table}
\end{savenotes}