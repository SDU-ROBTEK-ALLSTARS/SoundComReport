\section{Konklusion} 
Ved hj�lp af accelerometeret har det v�ret muligt at m�le udslag i sving og ved tilpasning af afkoblingskondensatorer har st�j kunne minimeres. Softwaren har gjort at en mere pr�cis detektering har kunne opn�s, idet fejlagtige m�linger er sorteret fra.

Optosensoren er brugt til at registrere, hvorn�r bilen passerer m�lstregen. Kredsl�bet til optosensoren har fungeret tilfredsstillende og ved hj�lp af en tidsforsinkelse har det kunne sikres, at den ikke registerer m�lstregen flere gange ved samme passage.

Kommunikationen med musesensoren (ADNS-9500) har vist sig at virke, men en periodisk fejl, hvor sensoren ikke l�ngere sender korrekt data, har gjort k�rsel efter opm�lt data umulig. Hele ideen bag opm�lt k�rsel bygger p�, at bilens position altid er kendt. Det har derfor ikke v�ret muligt at teste koden, der bruges i forbindelse med k�rsel efter indsamlet data, i praksis.

Registreringen af sving har dog vist sig at v�re en succes, p� trods af, at positionsv�rdierne ofte har v�ret ukorrekte. Kommunikationsprotokollen har liges� fungeret godt og har, som det var meningen, muliggjort datalogging fra f. eks. accelerometer og musesensor. Samtidig har kommunikationsprotokollen lettet debugging.

H-broen er testet og virker efter hensigten. Hvis der laves fuld reversering, reduceres bremsel�ngden betydeligt. 
 
Der er designet et SMD print til ADNS9500 sensoren og det har gjort en optimal montering under bilen mulig.

Samlet set kan det konkluderes, at bilen ikke kunne k�re efter logget data. Udover afstandsm�ling har opsamling af de �vrige data virket efter hensigten. 

