\subsection{Microcontroller}
I projektet benyttes en AVR ATmega32 microcontroller som samlingspunkt for sensorinput og til signalbehandling. Den styrer ligeledes motorhastighed og retning. Her beskrives kort, hvordan microcontrolleren er bygget op og hvilke funktioner - af dem der bliver udnyttet i dette projekt - den indeholder.

ATmega32 er en 8-bit microcontroller, der bygger p� RISC\footnote{Reduced Instruction Set Computing} designstrukturen, hvilket kort sagt betyder, at microcontrollerens instruktioner alle er forholdsvis simple og kun tager kort tid (mange kun 1 \textit{clock cycle\footnote{Tiden mellem �n stigende flanke til den n�ste stigende flanke p� et pulserende signal}}). ATmega32 microcontrolleren har 32kB programhukommelse, der ikke slettes, hvis str�mmen afbrydes og 2kB SRAM\footnote{Static random-access memory}, som ikke best�r hvis str�mmen afbrydes.

AVR microcontrolleren indeholder en lang r�kke periferiske enheder, hvoraf mange bliver brugt i forbindelse med kommunikation og input fra sensorer. Deriblandt er analog-til-digital konverteren, som kan ''overs�tte'' et analogt signal til en digital, bin�r, v�rdi i op til 10-bit opl�sning. Der benyttes ogs� en analog komparator, som funktionelt ligner et almindeligt komparatorkredsl�b (op-amp med positiv feedback). Der kan frit v�lges referencesp�nding til komparatoren - alts� den sp�nding indgangssignal sammenlignes med - mellem enten en intern 2,56 V eller en ekstern sp�nding p�f�rt en indgang.

SPI\footnote{Serial Peripheral Interface} og USART\footnote{Universal synchronous/asynchronous receiver and transmitter} er begge enheder til seriel kommunikation. De er ogs� i forvejen indbygget i ATmega32 og bliver beskrevet mere indg�ende i senere software afsnit.

