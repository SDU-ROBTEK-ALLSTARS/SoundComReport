\section{Programmering}
ATMega32 microcontrolleren, som Scalextric bilen er udstyret med, er samlingspunkt for al sensor input. Der f�s input fra musesensor, optocoupler, accelerometer og derudover skal motoren styres ud fra f�rn�vnte input. Alt dette er microcontrollerens opgave, hvilket alts� vil sige, at al aktivitet, der foreg�r, tager udgangspunkt i den software, der er skrevet til. I dette afsnit beskrives de forskellige programdeles form�l og funktion.

Den overordnede ide med softwaren er, at den skal v�re modul�r - forst�et p� den m�de, at det har skulle v�re let at �ndre bilens ''opf�rsel'' i forskellige situationer. Det har ledt til en opbygning af programmet, der er baseret p�, at der - enten manuelt eller ud fra sensorstimuli - kan skiftes mellem forskellige dele af programmet, ved at �ndre p� v�rdier i et register. �ndringen kan ske ligemeget hvor i programmet man befinder sig.

Den modul�re struktur af softwaren har lettet test af individuelle programstykker og har gjort, at den overordnede ide om k�rselsmetode har kunne f�lges. Der er s�ledes en del af softwaren, der tager sig af baneopm�ling og en der tager sig af k�rsel. Mellemliggende programniveauer s�tter rammen for hoveddelene. Enkelte dele, som f. eks. kommunikationsfunktionerne og ADNS sensordata opdatering, er konstant aktive.
