\section*{Opgave 4}\addcontentsline{toc}{section}{Opgave 4}\refstepcounter{section}
Vi har vektorfeltet
\[
	\mathbf{F} = 
	\begin{pmatrix}
		axy+3yz \\
		x^2+3xz+by^2z \\
		bxy+cy^3
	\end{pmatrix}
\]

hvor $a$, $b$ og $c$ er reelle konstanter. Det �nskes at bestemme integralet $I$
\[
	I = \int_{\mathcal{C}} \mathbf{F} \cdot d\mathbf{r}
\]
hvor $\mathcal{C}$ er en kurve fra $(0,1,-1)$ til $(2,1,1)$

\subsection*{a)}\addcontentsline{toc}{subsection}{a)}\refstepcounter{subsection}
Konstanterne $a$, $b$ og $c$ �nskes fundet, s�ledes at $I$ er uafh�ngig af kurvens vej. Vi ved, at hvis vektorfeltet $\mathbf{F}$ er konservativt, afh�nger $I$ netop \textit{ikke} af kurvens vej\footcite[866]{adamsessex}.

Hvis vektorfeltet $\mathbf{F}$ er konservativt eksisterer der en potentialefunktion $f$, s� $\mathbf{F} = \nabla f$. $I$ kan, idet vi kender $f$ ogs� skrives som\footcite[867]{adamsessex}
\[
	I = \int_{\mathcal{C}} \mathbf{F} \cdot d\mathbf{r} = \int_{\mathcal{C}} \nabla f \cdot d\mathbf{r} = f(P_1) - f(P_0)
\]
hvor kurven $\mathcal{C}$ g�r fra punktet $P_0$ til punktet $P_1$.

For at bestemme $f$ unders�ges ligheden $\mathbf{F} = \nabla f$
\begin{align*}
	\mathbf{F} &= \nabla f \implies \\
	F_1 \mathbf{i} + F_2 \mathbf{j} + F_3 \mathbf{k} &= \frac{\partial f}{\partial x} \mathbf{i} + \frac{\partial f}{\partial y} \mathbf{j} + \frac{\partial f}{\partial z} \mathbf{k}
\end{align*}

Hvert led kan s�ttes lig hinanden, s�
\[
	F_1 = \frac{\partial f}{\partial x}
\]
\[
	F_2 = \frac{\partial f}{\partial y}
\]
\[
	F_3 = \frac{\partial f}{\partial z}
\]

For at finde $f$ integreres den f�rste lighed med respekt til $x$ for at give et muligt udtryk for $f$. Derefter sammenlignes $f$, partielt differentieret mht. $y$ og $z$, for at opn� et mere fuldkomment udtryk for $f$. I l�bet af denne process kan konstanterne $a$, $b$ og $c$ samtidig bestemmes.
\begin{align*}
	\frac{\partial f}{\partial x} &= axy+3yz \implies\\
	f &= \int (axy+3yz) dx\\
	&= \frac{1}{2} ax^2y + 3xyz + g(y,z)
\end{align*}
Integrationskonstanten lader vi v�re en funktion $g$ af $y$ og $z$. Hvis $g(y,z)$ differentieres partielt mht. $x$ bliver den $0$ og ligheden g�lder.

Nu haves s�ledes et udtryk for $f$ som kan differentieres partielt mht. $y$
\begin{align*}
	\frac{\partial f}{\partial y} &= \frac{\partial}{\partial y} \left( \frac{1}{2} ax^2y + 3xyz + g(y,z) \right)\implies\\
	&= \frac{1}{2} ax^2 + 3xz + \frac{\partial g(y,z)}{\partial y}
\end{align*}

Dette udtryk sammenlignes med $F_2$ for at bestemme konstanter
\[
	x^2+3xz+by^2z = \frac{1}{2} ax^2 + 3xz + \frac{\partial \, g(y,z)}{\partial y}
\]

Vi ser at det m� betyde, at $a=2$ og 
\begin{align*}
	\frac{\partial \, g(y,z)}{\partial y} &= by^2z \implies\\
	g(y,z) &= \int (by^2z) dy\\
	&= \frac{1}{3}by^3z+h(z)
\end{align*}
Differentieres $g(y,z)$ mht. $y$ skal resultatet v�re 0, dvs. at $g$ h�jst kan v�re en funktion af $z$. Den kalder vi $h(z)$

Nu kan $g(y,z)$ inds�ttes i $f$
\[
	f = \frac{1}{2} ax^2y + 3xyz + \frac{1}{3}by^3z+h(z)
\]

Slutteligt differentieres $f$ partielt mht. $z$
\begin{align*}
	\frac{\partial f}{\partial z} &= \frac{\partial}{\partial z} \left( \frac{1}{2} ax^2y + 3xyz + \frac{1}{3}by^3z+h(z) \right)\\
	&= 3xy + \frac{1}{3}by^3 + \frac{\partial \, h(z)}{\partial z}
\end{align*}

og vi sammenligner med $F_3$
\[
	bxy+cy^3 = 3xy + \frac{1}{3}by^3 + \frac{\partial \, h(z)}{\partial z}
\]
hvor det ses, at $b=3$, $c=1$ og 
\begin{align*}
\frac{\partial \, h(z)}{\partial z} &= 0 \implies\\
h(z) &= k
\end{align*}
Differentieres en konstant $k$ bliver den $0$

Dermed kan et fuldkomment udtryk for potentialefunktionen $f$ opskrives idet konstanterne $h(z)$, $a$, $b$ og $c$ inds�ttes
\[
	f = x^2y + 3xyz + y^3z+k
\]

\subsection*{b)}\addcontentsline{toc}{subsection}{b)}\refstepcounter{subsection}
I denne opgave skal integralet $I$ bestemmes. I opgave 4a opstilledes f�lgende lighed 
\[
	I = \int_{\mathcal{C}} \mathbf{F} \cdot d\mathbf{r} = f(P_1) - f(P_0)
\]

Potentialefunktionen $f$ er allerede bestemt og v�rdierne for $a$, $b$ og $c$ er indsat, s� vi evaluerer blot udtrykket
\begin{align*}
	I &= f(2,1,1) - f(0,1,-1) \\
	&= (2^2 + 3(2) + 1) - (-1)\\
	&=12
\end{align*}
