\section*{Opgave 1}\addcontentsline{toc}{section}{Opgave 1}\refstepcounter{section}
Ved hj�lp af Laplacetransformation �nskes det at l�se differentialligningen

\begin{equation}
	y''(t)+2y'(t)-3y(t)=2t
	\label{eq:opg1}
\end{equation}

med begyndelsesbetingelserne
\begin{align}
	y(0)&=1\notag\\
	y'(0)&=2
	\label{eq:opg1_beting}
\end{align}

F�rste trin er, at finde Laplacetransformationen til hvert led i differentialligningen.
\begin{equation}
	\boxed
	{
		\mathcal{L}\{f^{(n)}\} = s^{n} \mathcal{L}\{f\} - \sum_{i=1}^{n} s^{(n-i)} f^{(i-1)}(0)
	}
	\label{eq:opg1_lapl_difflign}
\end{equation}

Laplacetransformationen til den $n$'te afledte af en funktion $f$ er givet\footcite[228]{kreyzig} i \eqref{eq:opg1_lapl_difflign}. Laplacetransformationen af $2t$ er givet i tabellen\footcite[224]{kreyzig} over funktioner og deres Laplacetransformeringer.
\begin{align*}
	\mathcal{L}\{y''(t)\}+2\mathcal{L}\{y'(t)\}-3\mathcal{L}\{y(t)\}&=\mathcal{L}\{2t\}\\
	&\implies \\
	s^{2} Y(s) - s y(0) - y'(0) +2 \left( s Y(s)-y(0) \right) -3 Y(s) &= \dfrac{2}{s^{2}}
\end{align*}

I det begyndelsesbetingelserne \eqref{eq:opg1_beting} inds�ttes, f�s
\begin{align*}
	s^{2} Y(s) - s y(0) - 2 +2 \left( s Y(s)-1 \right) -3 Y(s) &= \dfrac{2}{s^{2}}\\
	&\iff\\
	s^{2} Y(s)+ 2s Y(s) -3 Y(s) - s  - 4 &= \dfrac{2}{s^{2}}
\end{align*}

Der l�ses for $Y(s)$
\begin{align*}
	s^{2} Y(s)+ 2s Y(s) -3 Y(s) - s  - 4 &= \dfrac{2}{s^{2}}\\
	&\iff\\
	Y(s)\left(s^{2} + 2s - 3 \right) &= \dfrac{2}{s^{2}} + s + 4\\
	&\iff\\
	Y(s) &= \dfrac{\dfrac{2}{s^{2}} + s + 4}{s^{2} + 2s - 3}
\end{align*}

Nu kendes alts� Laplacetransformationen til $y(t)$. L�sningen til IVP'et f�s ved at tage den inverse transformering til $Y(s)$. Den letteste metode er nu at dele $Y(s)$ i partielle br�ker, s� der kommer nogle led, der kan transformeres tilbage vha. ''transformationstabellen''.

F�rst omskrives udtrykket for $Y(s)$ for at undg� at skulle opl�se i partielle br�ker to gange.
\begin{align*}
	Y(s) &= \dfrac{\dfrac{2}{s^{2}} + s + 4}{s^{2} + 2s - 3} \\
	&= \dfrac{2}{s^{2} (s-1) (s+3)} + \dfrac{s + 4}{(s-1) (s+3)}\\
	&= \dfrac{2 (s-1) (s+3) + (s+4) (s^{2} (s-1) (s+3))}{s^{2} \left((s-1) (s+3)\right)^{2}}\\
	&= \dfrac{s^3 + 4s^2 +2}{s^{2} (s-1) (s+3)}
\end{align*}

Den partielt opl�ste br�k bestemmes og der s�ttes p� f�lles br�kstreg: Der kan uden videre opl�ses i partielle br�ker idet graden af n�vneren er h�jere end graden af t�lleren.
\begin{align}
	Y(s) &= \dfrac{s^3 + 4s^2 +2}{s^{2} (s-1) (s+3)}\notag\\
	&= \dfrac{A}{s^2} + \dfrac{B}{s} + \dfrac{C}{s-1} + \dfrac{D}{s+3}\label{eq:opg1_part_brok}\\
	&= \dfrac{A(s-1)(s+3) + Bs(s-1)(s+3) + Cs^2(s+3) + Ds^2(s-1)}{s^2 (s-1) (s+3)}\notag
\end{align}

Med ens n�vnere m� t�llerne ogs� v�re lig med hinanden
\begin{align*}
	s^3 + 4s^2 +2 &= A(s-1)(s+3) + Bs(s-1)(s+3) + Cs^2(s+3) + Ds^2(s-1)\\
	&= As^2+2As-3A + Bs^3+2Bs^2-3Bs + Cs^3+3Cs^2 + Ds^3-Ds^2
\end{align*}

Det ses, at $A$ er den eneste konstant, der bliver st�ende n�r $s=0$
\begin{align*}
	2 &= -3A \iff \\
	A &= -\dfrac{2}{3}
\end{align*}

S�ttes $s=1$ bliver $C$ st�ende, da $(s-1)=0$
\begin{align*}
	s^3 + 4s^2 +2 &= Cs^3+3Cs^2 \Big|_{s=1} \implies \\
	C &= \dfrac{7}{4}
\end{align*}

S�ttes $s=-3$ bliver $(s+3)=0$, hvilket lader $D$ st� tilbage
\begin{align*}
	s^3 + 4s^2 +2 &= Ds^3-Ds^2 \Big|_{s=-3} \implies \\
	D &= -\dfrac{11}{36}
\end{align*}

For til sidst at bestemme $B$ isoleres,
\begin{align*}
	s^3 + 4s^2 +2 &= As^2+2As-3A + Bs^3+2Bs^2-3Bs + Cs^3+3Cs^2 + Ds^3-Ds^2 \iff \\
	B(s^3+2s^2-3s) &= s^3 + 4s^2 +2 - (As^2+2As-3A) - (Cs^3+3Cs^2) - (Ds^3-Ds^2) \iff \\
	B &= \dfrac{s^3+4s^2+2 -As^2-2As+3A -Cs^3-3Cs^2 -Ds^3+Ds^2}{s^3+2s^2-3s}
\end{align*}

og de �vrige konstanter $A$, $C$ og $D$ inds�ttes
\begin{align*}
	B &= \dfrac{s^3+4s^2+2 -\left(-\frac{2}{3}\right)s^2-2\left(-\frac{2}{3}\right)s+3\left(-\frac{2}{3}\right) -\frac{7}{4}s^3-3\cdot\frac{7}{4}s^2 -\left(-\frac{11}{36}\right)s^3+\left(-\frac{11}{36}\right)s^2}{s^3+2s^2-3s}\\	
	&= \dfrac{ -\frac{4}{9}s^3 -\frac{8}{9}s^2 +\frac{12}{9}s}{s^3+2s^2-3s}\\
	&= \dfrac{-\frac{4}{9} \big( s^3 + 2s^2 - 3s \big)}{s^3+2s^2-3s}\\
	&= -\dfrac{4}{9}
\end{align*}

Nu kan alle konstanter s�ttes ind i den partielt opl�ste br�k \eqref{eq:opg1_part_brok}
\begin{align*}
	Y(s) &= \dfrac{A}{s^2} + \dfrac{B}{s} + \dfrac{C}{s-1} + \dfrac{D}{s+3}\\
	&= \dfrac{-\frac{2}{3}}{s^2} + \dfrac{-\frac{4}{9}}{s} + \dfrac{\frac{7}{4}}{s-1} + \dfrac{-\frac{11}{36}}{s+3}
\end{align*}

Ved at tage den inverse Laplacetransformering (se tabel\footcite[224]{kreyzig}) kan l�sningen p� begyndelsesv�rdiproblemet findes
\begin{align*}
	y(t) &= -\dfrac{2 t}{3} -\dfrac{4}{9} +\dfrac{7 e^t}{4} -\dfrac{11 e^{-3t}}{36}
\end{align*}