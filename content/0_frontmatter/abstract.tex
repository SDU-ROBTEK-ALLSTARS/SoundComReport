\chapter*{Abstract}\addcontentsline{toc}{chapter}{Abstract}
This project is based on the given assignment to create a protocol stack which
use DTMF tones as information carriers, furthermore this has to be done through air
by speaker and microphone. The team agreed on pursuing a secondary objective which
was to develop an application programming interface for easy-to-use utilisation of
the protocol stack.

It is natural therefore to use the layers described in the OSI-model to define
the implementation of the protocol stack itself, and for the composition of the report.
Interdisciplinarity is a big concern for this project as networking- and datacommunication
theory is mixed with the basics of, digital signal processing, software development,
and C++ programming skills. Thus, this report will reflect these skills used to analyse given
problem statements and solving them.

Everything points in the direction of accomplishment of the objective as a fairly stable 
protocol stack was implemented, though it is not efficient compared to state of the art
data communication systems which exist today. In time of writing, the application programming
interface is still under development.

The result work as a proof of concept, that DTMF tones can be used as information carriers
when the data is transported in air. The significance of creating this proof have given a
great insight in topics regarding the basics of, networking and data communication,
digital signal processing, software development, and C++ programming.