\chapter{Encoding at the physical Layer}\label{app:physical_encode}

	
	
	
	
	
	The encoding scheme is handling the translation from frames into DTMF tones and visa versa. The encoding will be a
	two step process as frames are translated into a sequence of numbers which then are translated into a sequence of DTMF
	tones.
	
	A frame can be considered as a stream of bits, this stream is the actual data the frame consist of. This stream can be divided
	into 4 bit sequences where each number represent the value from zero to fifteen both included. These sixteen combinations
	match up with table \ref{tab:DTMF_mapping} and the four upper rows in table \ref{tab:newDTMF_mapping}. After dividing the 
	frame into nibbles\footnote{A 4 bit size} and each nibble is put into a list, this list now is a sequence of numbers. Each
	number is assigned to a combination of two tones according to table \ref{tab:newDTMF_mapping}. These frequencies can now be
	calculated based upon the number representing each nibble.
	
	Calculate low frequency:
	\begin{equation}f_{l} = \frac{number}{4},\end{equation}
	where $f_{l}$ is a whole number, meaning if the outcome is not whole, digits after the comma are discarded. Number
	represent a number from the sequence numbers representing the data stream of a frame.
	
	Calculate high frequency:
	\begin{equation}f_{h} = number\%4,\end{equation}
	where $f_{h}$ is a whole number from the modulus operator, this operator is used here to return the rest of the division.
	
	If low and high frequencies are arranged in two arrays as they are listed, one for low tones and one for the high tones,
	then $f_{l}$ and $f_{h}$ represent an index of a tone in the frequency array concerned.
	
	The generation of numbers from identified frequencies is also an easy calculation to carry out, as this would look like this:
	\begin{equation}number = f_{l} \cdot 4 + f_{h},\end{equation}
	This can be confirmed by table \ref{tab:newDTMF_mapping}.
	
	The latter calculations defines the encoding scheme between DTMF tones and numbers, which is one of the steps needed for
	the full encoding. The other step is to get a frame and translate it into a number sequence and likewise translating a
	sequence of numbers into a frame. At the sending site frames are split into nibbles and stuffed with control tones at relevant
	spots.
	
	An Example:
	

	
	As seen in table \ref{tab:physical_stuffed_frame} the frame itself is wrapped in control codes telling the software where
	the frame start and end. In between nibbles of the same bit combination a control code is added to indicate that two equal
	tones are going to be played right after another as this frame is transmitted.