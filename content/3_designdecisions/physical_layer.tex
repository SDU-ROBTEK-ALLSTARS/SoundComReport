\section{Physical Layer}

	\subsection{Overview}
	\todo[inline]{Fjerne overview overskriften? Efter section{}-heading b�r derv�re en ''introduktion'' til det, overskriften d�kker over, s� det ville jo passe fint}
	This chapter will contain an analysis of how its possible to let DTMF tones
	enter into data communication through air.
	
	This analysis will be based on a minimal understanding of the physical aspects
	of sound waves traveling through air, and on how to make an software
	implementation of the physical layer which is an part of the OSI model.
	
	After ward this will be tested \todo{F�rst i senere chapter?} for stability issues and performance capabilities.
	
	The results will be discussed and this will lead to several inferences which will
	tell how the goal of using DTMF as information carrier was best achieved.
	
	\nomenclature{DTMF}{Dual-tone multi-frequency}
	\subsection{Physical Layer}
	In data communication and networking theory the physical layer is the lowermost
	layer \todo{Paraph} in the OSI model. The OSI model consist of seven different layers which all
	defines specific tasks for data communication can appear throughout the layers.
	
	The task of the physical layer, is getting a physical signal translated into a data, which
	then can be delivered as a frame to the above lying layer, which is the data link layer.
	All this is happening at the receiving side of the physical layer. Communication through
	the physical layer would not be usable if the physical layer is not able to send any signal,
	so the physical layer would obviously have to define a set of tasks to translate frames from
	the data link layer into signals that can be transmitted. Thus, the physical layer defines
	electrical and physical framework for receiving and transmitting signals through the media,
	and furthermore it defines encoding/decoding and alignment schemes for translating
	frames into signals and visa versa.
	
	This project requires the use of DTMF tones as information carrier, and furthermore it is
	required that transmission are broadcast through the air. These requirements decides some
	of the properties of the physical layer, the media became air and the signals that will be
	translated are DTMF tones.
	
		\subsubsection{How can DTMF enter into data communication through Air?}
		DTMF is an abbreviation for Dual Tone Multiple Frequency. It is a system of tones that are used by
		telephones when dialing a number. The system is an arrangement of four low tones and four high tones,
		it is arranged in a four-by-four matrix (see Table \ref{tab:DTMF_mapping})which gives the system sixteen combinations.
		
		The idea is that these sixteen combinations formed by the DTMF matrix is used in some kind of data
		communication between two or more computers. To let DTMF tones enter into data communication \todo{Paraph} we have
		to apply the property of  waves to carry bits. This can be done by letting each entry in the matrix
		consist of four bits. Four bits gives sixteen combinations which each can be assigned to an entry in
		the matrix. Below is shown the DTMF map which will be used in the physical layer for encoding and decoding.
		
		\begin{table}[htb]
			\begin{center}
				\begin{tabular}{c|c c c c}
				 & 1209 Hz & 1336 Hz & 1477 Hz & 1633 Hz \\
				\hline
				697 Hz & 0000 & 0001 & 0010 & 0011 \\
				770 Hz & 0100 & 0101 & 0110 & 0111 \\
				852 Hz & 1000 & 1001 & 1010 & 1011 \\
				941 Hz & 1100 & 1101 & 1110 & 1111 \\
				\end{tabular}
			\end{center}
			\caption{Table of the matrix with the bit combinations assigned to each entry.}
			\label{tab:DTMF_mapping}
		\end{table}
		
		Now by letting one computer play the tones through a speaker and another computer record it from
		a microphone then data is transmitted through air. The exchange of data is essentially an exchange
		of information and it is not satisfying exchanging information of the size of four bits alone,
		cause this would make the system inefficient. The system need to be able to transmit a sequence
		of tones matching a given bit pattern. The number of tones played each second will determine the
		bit rate of the communication. The bit rate will of course be dependent on how fast the sampling and
		recognition of tones can be handled at the receiving end.
		
		\subsubsection{Physics}
		As sound waves are considered to be a mechanical waves, the physics of mechanical waves therefore
		applies to the sound of DTMF tones. Intentionally this section will not give at deeper understanding
		of the physics, but just state some of the constants involved. This is done to state delays in the
		communication system. Delays should not be a problem as the physical layer will implement a
		synchronization method between sender and receiver site.
		
		Sound waves travels at a velocity of
		\begin{equation}v_{s} = 343 \frac{m}{s}\end{equation}
		in air with the temperature of 20 degrees Celsius.
		
		\subsubsection{Port Audio Interface}
		It was decided in the initial phase of the project \todo{Ref} to use Port Audio
		as the interface between the developed API and the sound card in a computer. Several points
		was taken into account for this decision, it has to be easy to use, it has access to the recorded
		samples and output sample buffer, and it has to be cross platform. Port Audio is satisfying all these needs.

		The interface that is developed for this project will implement Port Audio in its simplest
		form along side with a buffer system for incoming samples and outgoing samples.
		
		For the receiving part the buffer that will be implemented is a FILO\footnote{First In Last Out}
		\nomenclature{FILO}{First In Last Out}
		buffer which hold identifiers of a certain DTMF tone and each time a DTFM tone is detected \todo{vides det } it is added
		to the buffer. The size of this buffer will be the size of a data link layer frame plus the size of some
		preamble which will indicate a transmission is about to start.
		
		The buffer which will support the outgoing data will be build as a two dimensional buffer. This
		is done because as outgoing data needs to be a sequence of DTMF tones, but in discrete time each
		tone will be a sequence of samples. This leads to the two dimensional data structure of the size N times M.
		
		\begin{table}[htb]
			\begin{center}
				\begin{tabular}{c|c c c c c c c}
				 & n = 0 & n = 1 & n = 2 & n = 3 & n = 4 & \ldots & n = N - 1 \\
				\hline
				Tone 1 & 1.0 & 0.83 & 0.3 & 0.01 & -0.5 & \ldots & -1.0 \\
				Tone 2 & 0.0 & -0.83 & 0.43 & & & & \\
				Tone 3 &  &  &  &  &  &  & \\
				\vdots &  &  &  &  &  & $\ddots$ & \\
				Tone M &  &  &  &  &  &  & \\
				\end{tabular}
			\end{center}
			\caption{Table of how the two dimensional buffer could look like at the sending site of the interface.}
			\label{tab:2d_buffer}
		\end{table}
		
		\subsubsection{Goertzel Algorithm}
		To be able to detect if tones have been transmitted some algorithm for detection of tones have to be implemented.
		For this purpose the Goertzel algorithm is used, this algorithm has the ability to detect if a signal contain a
		specific frequency. Other algorithms exist which would provide the same information but those would be
		much more expensive regarding the computational complexity. Those other algorithms are called Discrete Fourier Transform (DFT)
		and the other is called Fast Fourier Transform (FFT) which is a more efficient way to obtain the same information
		as the DFT algorithm. The computational complexity of DFT, according to Li Tan \cite[124]{DSP}, \todo{Indsat ref. - hvad m�de skal det g�res p�? Bare vi g�r det samme} is $N^2$, where $N$ is the number of samples.
		For FFT (see Li Tan \cite[124]{DSP}) the computational complexity is calculated to be $\frac{N}{2}\cdot \log_{2}(N)$,
		where $N$ is the number of samples.
		\nomenclature{DFT}{Discrete Fourier Transform}
		\nomenclature{FFT}{Fast Fourier Transform}
		
		As the detection of tones has to occur as fast as possible it is desired to lower the cost of cpu power by using 
		an algorithm which has the least computational complexity. The Goertzel algorithm therefore suit this need very well.
		The reason for this is that with a few pre-calculated constants and $N$ iterations over $N$ samples, \todo{En iteration = en multiplikation og en addition?} a value is returned
		which indicate if a specific frequency is present in the incoming signal.
		
		Essentially the Goertzel algorithm is a second order IIR filter which is dependent on current input and previous
		output, the filter is given \todo{Ref?} as the sequence shown below:
		\begin{equation}y(n) = x(n) + 2\cdot \cos(2\pi \cdot f_{0})\cdot y(n - 1) - y(n - 2),\end{equation}
		where $f_{0}$ is the frequency of interest.
		
		By Z-transform the following is obtained:
		\begin{equation}H(z) = \frac{Y(z)}{X(z)} = \frac{1}{1 - 2\cdot \cos(2\pi \cdot f_{0})\cdot z^{-1} + z^{-2}}\end{equation}
		
		As the above equation show, it is a second order IIR filter. This then have to be implemented as a direct form-II structure
		where the point W \todo{Er point W p� billedet? :))} is of interest.
		\nomenclature{IIR}{Infinite Impulse Response}
		
		\textbf{\textit{\underline{Picture of direct form-II here!}}}
		
		The point W will be used for calculating the frequency response at a specific frequency of interest. As this is taking
		place in discrete time, an expression of the frequency of interest is needed. In discrete time the frequency spectrum
		is divided into frequency bins, the size of these bins is determined by the number of samples and the sample rate.
		Obtaining the $k$'th bin can be done as shown below:
		\begin{equation}k = \frac{f_{0}}{f_{s}}\cdot N\end{equation}
		where $f_{0}$ is the frequency of interest, $f_{s}$ is the sampling frequency, and $N$ is the number of samples.
		
		Now it is possible to calculate the filter coefficients, which in the case with the Goertzel algorithm reduces 
		to one single coefficient. This coefficient have to be calculated for each tone that want to be identified, but can 
		be calculated in advance because its only dependent on the frequency of interest.
		
		The coefficient is calculated with:
		\begin{equation}c = 2\cdot \cos(2\pi \cdot \frac{k}{f_{s}})\end{equation}
		
		Now that all constants can be calculated in advance, the detection system only have to calculate the filter output, calculate
		the magnitude of the frequency and then it is able to decide based on the result if the frequency exist in a incoming signal.
		As detection of DTMF tones is needed for this application it needs to determine if two specific frequencies is present at the
		same time. But this wont be much of a problem as the constants for each frequency can be calculated in advance.
		The algorithm are implemented as a direct form-II structure, and need to iterate over the array of collected samples to be
		able to detect if a given frequency is present in a signal.
		
		The computational complexity can therefore be written as one multiplication plus two additions per iteration per tone.

		This leads to:
		\begin{equation}\mathrm{complexity} = 3\cdot N\cdot M\end{equation}
		where N is the number of samples and M is the number of tones.
		
		For one tone detection over 205 samples this is around 600 calculations, for detection of 8 tones simultaneously the equation
		above result in around 5000 calculation, and these real calculations where as for DFT and FFT the computational complexity 
		is much higher and the calculations are carried out with complex numbers.
		
		This is the reason for choosing the Goertzel algorithm for detection of frequencies.
		
		\subsubsection{Synchronization}
		To be able to identify if the recorded sound leads to the reception of data, a frame from the data link
		layer need to be wrapped in a header which consist of a preamble and possible a tail to indicate transmission has ended.
		As a stream of data is emulated by sending a sequence of DTMF tones, what happens if the data stream gives rise to two identical
		DTFM tones in a row, \sout{how is this handled}? \todo{Att}
		
		There is several possibilities to encounter \todo{\xout{en}counter} this problem. The first solution could be to stuff the frame with a given bit
		pattern to indicate that on each side of stuffed bits is identical bit patterns. This solution would only affect the data 
		link layer implementation.
		
		The second solution is to introduce a new frequency so another DTMF tone would appear outside the four by four matrix.
		This tone could then be stuffed into the sequence of DTMF tones to indicate that on each side of this tone is two identical
		tones. This solution affects the physical layer implementation.
		
		The third option is to make a little silence between each tone to indicate a new tone is about to be played at the sending computer.
		
		To avoid tampering with the data link layer as the synchronization is a part of the physical layer option number one is not commendable. Option
		number three on the other hand would require an implementation of a silence detection system. \todo{When numbering them, maybe the numbers should be included in the describing text}
		
		Option number two requires that a single frequency is added to the table of tones, and the detection of this tone is added to the implemented
		detection system. Every time this tone is detected, it is not added to frames that is about to be received, but it will indicate a that two
		identical bit stream are received one after another.
		
		Option number two would be recommended as this solution is independent of detection of silence, which essentially would be an detection of
		white noise. Furthermore it seems that the implementation of this solution would require a minimum of effort to work.		
		
		\subsubsection{Encoding scheme}
		
		
		\subsubsection{Software flow}
		
		\subsubsection{? What Else ?}
		
	\subsection{Test}
	\todo[inline]{Til experiments chapter }
	\subsection{Results}
	
	\subsection{Interpretation}
	
	\subsection{Conclusion}