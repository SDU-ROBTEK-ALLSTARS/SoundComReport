\section{Secondary requirements}\label{sec:secondary_requirements}
The team decided on expanding the list of requirements, this was done because guidelines regarding the tools for planning, documentation, and development of this project was needed. This was done because along with the theory it was desired to get a basic knowledge of these tools as they ease the administration of maintaining documents and software when several people are working at the same project at the same time.

\begin{itemize}
\item Easy-to-use application programming interface to be developed as distributed software.
\item The networking library should enable multi-point communication.
\item The networking library has to be cross platform.
\item The report itself is typeset in \LaTeX\ and is written in English.
\item Google\footnote{\url{https://docs.google.com/}} docs is used for internal documentation, minutes of meetings, etc.
\item Google\footnote{\url{https://www.google.com/calendar/}} calendar is used for time management.
\item Trello\footnote{\url{https://trello.com/}} is used as the planning tool for deadlines, to do lists, etc.
\item Git and GitHub\footnote{\url{https://github.com/}} is used as version control system for the report and the software. The repository, containing full history of the commits to the project, can be cloned from \small{\url{ https://github.com/SDU-ROBTEK-3SEM-E11-G1/SoundComCode.git}}
\end{itemize}

%
%Low level protocol til at ligge under styresystemet - tcp/ip alternativ.
%Stripped HTML client/server - g� p� tekst baseret net over DTMF
%Socks over DTMF
%Generisk C++ lib til kommunikation over DTMF
%Maksimal hastighed
%Minimal latency
%Multipunkt protocol
%Multi OS system
%Lang distance protocol
%Automatisk (gen)forhandling af hastighed for at opn� maksimal hastighed under alle forhold.
%Chat klient
%S�nke slagskib
%Send fil
%Hvordan lydstyrken skal bestemmes, konstant er selv-justerbar?