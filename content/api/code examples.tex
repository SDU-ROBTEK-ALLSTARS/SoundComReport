\subsection{Usage example}
\subsubsection{Creating instance}
\begin{lstlisting}[float=htb,language={[ANSI]C++},caption={Creating instance example},label=CreatingInstanceEx]
int main()
{
unsigned char myAddress = 1; // My address is 1
bool hasToken = true; // When starting I have the token. Only one node should have this flag set. When connection to an already running DTMF network this should be set to false.

DtmfApi *api = new DtmfApi(myAddress, hasToken);
// DTMF API is now running, but no ports are enables, so it can't receive any data.
(...)
\end{lstlisting}

\subsubsection{Adding a port callback example}
(This example assumes that an instance of API already has been created. See example \ref{CreatingInstanceEx})
\begin{lstlisting}[float=htb,language={[ANSI]C++},caption={Adding a port callback example},label=CreatingCallbackEx]
// Declare callback class
class MyPort1 : public DtmfCallback
{
	virtual void callbackMethod(DtmfInMessage * message)
	{
    // This function is called everytime data arrives. The incoming message only exist inside this method, so it's recommended to copy all needed information before exiting the function.
    // Fetch data from DTMF class to local memory (Assuming we have a dataContainer available in the MyPort1 class)
    m.getData(this->dataContainer*, 0, m.getMessageLength());
    
     // Data is now stored locally, return.
    return;
	}
}
// Make an instance of the class
MyPort1 * myPort1 = new MyPort1();
// Add the object as callback for port 1.
api->servicePort('1',myPort1);
\end{lstlisting}


\subsubsection{Sending data}
(This example assumes that an instance of API already has been created. See example \ref{CreatingInstanceEx})
The library support two ways of sending a message. Each with different advantages.
First method - the fast way:
\begin{lstlisting}[float=htb,language={[ANSI]C++},caption={Sending data example 1},label=SendingDataEx1]
unsigned char rcvAddress = '2';
unsigned char rcvPort = '1';
unsigned char * helloWorld = (unsigned char*)\"HelloWorld\";
api->sendMessage(rcvAddress, rcvPort, helloWorld, std::strlen(helloWorld));
\end{lstlisting}

Second method - building a custom message:
\begin{lstlisting}[float=htb,language={[ANSI]C++},caption={Sending data example 2},label=SendingDataEx2]
unsigned char rcvAddress = '2';
unsigned char rcvPort = '1';
unsigned char * hello = (unsigned char*)\"Hello\";
unsigned char * world = (unsigned char*)\"World\";
DtmfOutMessage message = api->newMessage(); // Create an empty message
message.setData(hello, 0, std::strlen(hello)); // Add hello to message
message.setData(world, std::strlen(hello), std::strlen(world)); // Append the world to message
message.send();
\end{lstlisting}


\subsubsection{Deleting instance}
\begin{lstlisting}[float=htb,language={[ANSI]C++},caption={Deleting instance example},label=DeletingInstanceEx]
delete(api); // DTMF API is now shutting down. This operation may take a while.
(...)
\end{lstlisting}