\chapter{API}
\nomenclature{API}{Application programming interface}
The application programming interface (API) layer is designed with the goal of presenting an easy-to-use interface for the Dtmf library. The following chapter goes trough some of the considerations and tequniques used in the api layer design.

(The message buffers should be included in this section)

%Threading is used to avoid stalling the front end application when sending and receiving data. /* To be rewritten */

% Notes about writing the API documentation.

% Introduction
%  Short, about what to read in the next few pages.
% [Userbility] Considerations ( + What did i decide during the considerations)
%  Easy usage for the end user
%  Threads to minimize dirsturbance of the users application
%  Using callbacks to receive data on ports
% Implementation (How was it implemented)
%  Evalutation of public api methods
%  Short usage example [ Send | Receive / Setting up callback/port ]

% Threading
% - Thread safe buffers +  a little about race conditions

% May be used / refered
% - Message types
% - Life of a message
% - Dependency diagram

% Drawing showing the API layer as an interface with a cloud representing the underling layers

% Test suite



% Limitations of the protocol? Max numbers of stations, number of ports, max datasize.

% DtmfInMessage::getData not implemented


% DtmfInMessage - who can create it?





% Api documentation
% * Remember to be compatible with General Concepts > Facade class
