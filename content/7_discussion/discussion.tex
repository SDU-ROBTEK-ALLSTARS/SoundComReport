\chapter{Discussion}\label{chap:discussion}
The main idea about implementing the DTMFBTINPL has been to find an approach
where each member of the team was able to dig deeply into the essence of the
professions while still making it easy to compile the different parts of the
project into a coherent solution.

The chosen approach was to define clean interfaces between the software layers
in the form of predefined buffers. This has proven to be a powerful tool, since
each layer has been debugged and tested individually. Also a lot of potential
problems about the overall flow- and software control was avoided, since all
methods and loops were connected in the backbone program.

The architecture has provided a certainty of maximum reuse of
code and have efficiently avoided unnecessary coupling in the software. Because
of the bold ambition to create a compiled library for other developers to use, the problem reached a
size, where efficient programming was imperative. Time was also an issue and
again the procedure has proven most beneficial because each team member has been
able to work and test independent of the others, thereby avoiding time wasted in
waiting for other parts to finish.

The respective layers have been developed using an iterative approach, where
the individual team member has worked on his part and then presented it to the rest
of the group either by internal documentation on the project wiki, input to the
report, demonstration programs or by giving short lectures in the subject. This
way the team has been able to work efficiently while still pulling together.

The distribution of team roles has been a challenge, mostly because it implies a
new way of thinking in behaviour and responsibilities instead of duties and
fields of expertise. It was an informed choice to select the team roles from a
point of learning and not a point of experience. This decision ensured that
all members started on equal feet, but also imbued some difficulties of breaking
old patterns. It has provided some serenity to the work, knowing that one does
not have to be mindful of the entire project, trusting that other team members
will do their part.

Professionally the project has given hands on experience of the OSI layers
taught in data communication and a much greater understanding of the material.
Developing a large software system leads to considering the UML language and
advantages of object oriented programming, supporting the lectures in software
development. Since all members of the team have contributed to the source code,
a much greater experience in C++ programming has been obtained. Finally the
analysis of the problem and to some extend implementing the tone detection
system has lead to a greater understanding of digital signal processing.