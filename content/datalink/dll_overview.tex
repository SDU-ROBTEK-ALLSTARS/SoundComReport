\section{Overview and considerations}\label{sec:dll_theory}

There are several subjects to examine before designing the data link layer.
First of all the requirements must be identified. 
\begin{itemize}
	\item Short frames
	\item Error detection
	\item Error correction
	\item Pipelining
	\item Multipoint
\end{itemize}

\subsection{Encoding}
The first thing to consider is the encoding of signals. Since there are sixteen
different DTMF combinations, each tone can carry four bits.

The only type of error to consider is the situation where a tone is
misinterpreted, which leads to a four bit burst error (reference). Therefore the
system should be designed specifically to detect errors of this type. Since the media
is considered to be very noisy, transmissions should also be kept as short as
possible.

A two dimensional parity check will be able to detect burst errors of the
proposed size, so this is the choice. Implementing the parity check as a
four-by-four matrix will make it possible to transfer two bytes with a frame of
three bytes.

Example: We want to transmit the two bytes 0011 0101 and 0101 1110. The data
link layer puts these in a four by four matrix and calculates the parity bits by
adding the rows and columns as shown in table \ref{tab:two_dimensional_parity_check}.

\begin{table}[htb]
	\centering
	\begin{tabular}{c|c}
	0011 & 0 \\
	0101 & 0 \\
	0101 & 0 \\
	1110 & 1 \\
	\hline
	1101 & \\
	\end{tabular}
	\caption{Two-dimensional parity check}
	\label{tab:two_dimensional_parity_check}
\end{table}

Instead of increasing the size of each row by one to contain the parity bit as normally done, the parity bits are transmitted together as a redundant byte. In the case of this example the transmission would be as shown in table \ref{tab:bytes_to_be_transmitted}.

\begin{table}[htb]
	\centering
	\begin{tabular}{c|c|c|c|c|c}
	0011 & 0101 & 0101 & 1110 & 0001 & 1101 \\
	\end{tabular}
	\caption{Bytes to be transmitted}
	\label{tab:bytes_to_be_transmitted}
\end{table}

Each four bit nipple is now transmitted as a DTMF-tone. Should one of the tones
be misinterpreted, the receiving data link layer would get a mismatch of the
parity bits. An example of this is shown in table \ref{tab:failed_parity_check}.

\begin{table}[htb]
	\centering
	\begin{tabular}{c|c}
	0011 & 0 \\
	0101 & 0 \\
	0000 & 0 \\
	1110 & 1 \\
	\hline
	1000 & \\
	\end{tabular}
	\caption{Failed parity check}
	\label{tab:failed_parity_check}
\end{table}

This will lead to the frame being discarded. Though in some cases it might be
possible to correct the error and find the original nibble, this is not
recommended, since more than one tone might be corrupted. Furthermore it
complicates the data link layer significantly, and the gain is low since the
frame could be re-transmitted in six extra tones. Errors in the redundant byte
will also lead to discarding the frame.

\subsection{Flow control}
The next aspect to consider is flow control. Since the DTMF-system
cannot be used as full-duplex, piggybacking (reference) is impossible. This
means that eventually the receiver will have to reply. This reply will also
be of six-tones to take advantage of the parity system, and therefore it might
as well contain information about witch frames to resend. In other words a
selective repeat system is preferred. To introduce a selective repeat system,
additional redundancy is needed.

Pipelining must be considered to increase the speed of the system and is
implemented as a three bit sequence number. This limits redundancy to a minimum
while still benefiting from pipelining. The sender transmits eight frames
(forty eight tones) and then waits for the receiver to reply with one frame
(six tones). Should the receiver not respond or should the reply be lost, the
sender will automatically re-transmit after a while.

\subsection{Frame design}\label{dll_frame_design}
The physical layer will provide a starting point for each
transmission. If this was not the case,
additional flags would be needed in between the frames, leading to the
need of stuffing (reference) and to additional redundancy.

A frame size of three bytes is preferred for the proposed parity system. This
means preferably transmitting one byte per frame. A smaller payload would mean
increased complexity and lesser efficiency.

Normally bytes are transmitted in chunks of eight, but since the length of
datagrams can vary, the sender must have a way of telling the receiver to process smaller
chunks. This introduces the need for an End Of Transmission flag (EOT).

The next thing to consider is multipoint. There are three options: A token
network (reference), a time division network (reference) or a code division
network (reference). Time division requires a level of timing the interface layer
is unable to deliver. Code division leads to the need for larger frames or if implemented with the
proposed frame size, a lot of unused frames in a small network. This leaves us
with a token passing network, so this is the choice.

The selective repeat system and the token network introduces the need for
different frame types. The proposed type field will consist of two bits, controlling the token and indicating frame types at the same time. Table \ref{tab:protocol_for_type_field} shows the meaning of the two bits.

\begin{table}[htb]
	\centering
	\begin{tabular}{|c|ll|}
		\hline
		00 & Has no token & Reply from receiver \\
		\hline
		01 & Has no token & Passes token to addressed station \\
		\hline
		10 & Has token & Accepts token \\
		\hline
		11 & Has token & Data frame for addressed station \\
		\hline
	\end{tabular}
	\caption{Protocol for type field}
	\label{tab:protocol_for_type_field}
\end{table}

%  from receiver

In reply frames each bit of the data byte corresponds to a
sequence number and has value 1 for accepted and 0 for resend.

Since two bits are reserved for type, three for sequence number and one for flag,
the remaining two bits will control the addressing. Both can be used
for identifying the receiver, since info about sender is not needed at this
level. Thereby the protocol allows networks of up to four stations.

Token passing is controlled entirely by the data link layer. When the token is
offered, there is a window of response time, wherein the station must reply by accepting
the token. If there is no reply during the window, the token is offered to the
next station. 

This leads to the following format of a frame: 

\begin{table}[htb]
 \centering
 \begin{tabular}{|c|c|c|c|c|c|}
  \hline
  type & address & sequence & EOT-flag & data & parity \\
  2 bits & 2 bits & 3 bits & 1 bit & 8 bits & 8 bits\\
  \hline
 \end{tabular}
 \caption{Final frame format}
 \label{tab:final_frame_format}
\end{table}
% \begin{tabular*}{0.75\textwidth}{@{\extracolsep{\fill}}|c|c|c|c|c|c|}