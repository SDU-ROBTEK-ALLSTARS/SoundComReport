\chapter{Conclusion}\label{chap:conclusion}
Questions to be answered:

To what extend was it possible to make an API that implements a DTMF based
network?
How did the system perform during test? 
Are there things that should be testet further?

To what extend does the physical layer work? 
The physical layer works like a charm, according to table \ref{tab:exp_phys_speaker} a bit rate around $110\sfrac{bit}{s}$ can be reached without compromising the reliability too much. Several optimisation possibilities are available, but not needed as stability is not an issue.

  To what extend does the data link layer work? 
The data link layer is able to process a packet and enclose it in a sequence of
frames to be sent. It is also able to receive frames and to process them into at
packet to be delivered to the transport layer. The data link layer is also able
to control a token based network in a way that allows collision free half duplex
communication. 

  How did the data link layer perform during test? The data link
layer performed well under test and handles communication with up to sixty
percent errors, though this amount of errors makes it very slow. With more than
sixty percent errors all time goes into contol talk and almost no frames are
transmitted.

  What could make it even better?
If it is possible to find a way of using a sliding window instead of the eight
byte list currently implemented. There is currently a lot of transmission time
being wasted in sending lists shorter than eight entries. Other changes could be
considered to handle cases where a crucial frame is lost. For example the case
where a reply is lost which results in some waiting time and the entire list
being resend.

To what extend does the transport layer work? 
How did the transport layer perform during test?
What could make it even better?

%To what extend does the API layer work? 
The API layer has only been tested for functionality as it was not possible to make any performance test on this layer. It is able to create user messages and pass these on to the correct buffers without making threading problems. The messages can be picked up by the backbone and send to the transport. It has also been tested that the API layer is able to receive data from the backbone, and call the user defined callback method.
%Further improvements on the API layer depends greatly on for which specific application the library is to be used. 
%How did the API layer perform during test?
%No performance test was performed on the API layer.
The API layer could be improved by extending the feedback functionality such as notification about the quality of the link and whether a message has been send or still is queued. These functionalities would of cause require the protocol layers to provide these informations to the API layer.


To what extend does the Backbone work? 
The backbone class works as expected. It is able to dispatch work to the correct layer objects, based on the state of the internal buffers and layers.

How did the backbone perform during test?
The backbone has not been tested, as it is meaningless to test it without the actual layers combined. Since the final assembly has not been completed at the time of writing, any backbone tests are impossible. However the backbone has been debugged and everything seems to be functioning properly.

What could make it even better?
The input and output stacks could have their own backbone thread, inorder to simplify the logic of which buffer to check.
The communication to the physical layer could be handled through an actual queue with a mutex on the last element, instead of a ringbuffer. The logic to detect whether the network layer and datalink layer have room to perform their actions needs to be improved.


How did the test function work?
When used the test function worked as planned, and where adapted to the tested layer. 
How much was it used?
Some used the test function to a great exstent, unfortunaly not all tests were performed with the test function. This could have been solved by being more clear from start about how the layers were to be tested . 

To make the test function more efficient more options for testing could have been added.
